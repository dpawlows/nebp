\documentclass[12pt]{article}

%%%%%%%%%%%%%%%%%%%%%%%%%%%%%%%%%%
%%
%% Preamble
%%
%%%%%%%%%%%%%%%%%%%%%%%%%%%%%%%%%%

\usepackage{epsfig}
\usepackage{graphics}
% \usepackage{color}
\usepackage[square]{natbib}
%\usepackage{helvet}
%\usepackage{pslatex}
%\usepackage{times}
%\usepackage{palatcm}
\usepackage[sc]{mathpazo}
\usepackage{wrapfig}
\usepackage{pawlowski}
\usepackage{pdfpages}
%\usepackage{sectsty}


%\setcounter{secnumdepth}{0}

\setlength{\oddsidemargin}{0 in}
\setlength{\evensidemargin}{0 in}
\setlength{\topmargin}{-.20 in}
\setlength{\textwidth}{6.5 in}
\setlength{\textheight}{9.0 in}

%\renewcommand{\sfdefault}{phv}
%\renewcommand{\rmdefault}{phv}


\begin{document}

\Large
\begin{center}
The EMU Atmospheric Physics EXploration (EMU APEX) Program\\
\large
\vspace{2cm}
Nationwide Eclipse Ballooning Project\\
{\bf Engineering Track}
\vspace{1in}

\normalsize
\begin{table}[h]
\centering

  \begin{tabular}{ll}
  David Pawlowski&Thomas Kovacs\\
  Professor of Computational Physics&Professor of Meteorology\\
    Physics and Astronomy Dept.&Geography and Geology Dept.\\
  dpawlows@emich.edu&tkovacs@emich.edu\\
  240D Strong Hall&140 H Strong Hall\\
  734-709-0311&734-487-8591

  \end{tabular}
\end{table}


Eastern Michigan University\\
 R2 Research University\\
Ypsilanti, MI 48197
\end{center}

\small\normalsize
\newpage
\tableofcontents

\newpage
\section{Team Mentoring Approach Plan}
\label{intro}
The primary objective of this project is to form an interdisciplinary
high altitude weather balloon (HAB) program at Eastern Michigan University (EMU): EMU APEX.
The first cohort of students will design, construct, and launch a
high altitude weather balloon payload
capable of taking measurements during 2 upcoming solar eclipse events and help
establish a long-term stratospheric ballooning program. We see this as an opportunity to build a sense of
community between the students across the College of Arts and Sciences at EMU and envision a program
that leverages the expertise of upper level science students from the Department of Physics and Astronomy
to work alongside students from introductory courses (initially, ESSC101:
Introduction to Weather and Forecasting). By participating in the {\bf Engineering Track}, we look to
provide the participating students an opportunity to get their hands dirty by taking part in the
engineering design process and designing and building scientific instrumentation capable of making
meaningful contributions to the understanding of our atmosphere.

\subsection{Project Overview and Timeline}
To start the EMU APEX program, students will participate in the NEBP and prepare for
and participate in the two eclipse campaigns. We will initially recruit students from 2 student populations and incorporate
activities related to the HAB project in existing courses: PHY420- Physics
Capstone, ESSC101- Introduction to Weather and Forecasting, and PHY379- Special Topics in Physics.
As faculty mentors,
we do not have specific science goals that the students must meet for the eclipses. Rather,
{\bf the students will establish the science goals as a team }with guidance from the faculty.
Our only science requirement will be that the payload be designed to study how
atmospheric structure and/or dynamics might be altered during a solar eclipse.

% Additionally, most of the PHY420 students will graduate in May or December 2023, so in order to ensure continuity
% with the project, we will offer a special topics course (PHY379) that Junior and Sophomore
% level participants will take. This course is necessary due to prerequiste requirements and other
% course requirements for PHY420 students.

Prof. Pawlowski is the instructor for PHY420. As part of this course, students are tasked with completing
a semester long research project; incorporating stratospheric ballooning into it is a
natural fit. Prof. Kovacs is the instructor for ESSC101, which is a course that draws a diverse group of students looking to complete a general education requirement. Students often use weather balloon data to forecast precipitation type, calculate temperature lapse rates and determine static stability for severe storm forecasting. During the Winter 2023 semester, under the guidance of PHY420 students,
 ESSC101 and PHY379 students will learn about the design, construction and operation
of a high altitude weather balloon in order to prepare them for taking ownership of the
ballooning program after the semester ends. ESSC101 and PHY379 students will
participate in the design and build of the initial payload with assistance from the PHY420 students.

Under the guidance of Profs. Pawlowski and Kovacs, all students will work collaboratively to establish important milestones and to identify
and assign roles within the project based on NEBP website recommendations and requirements.
The team will collectively decide on the specific scientific payload to be designed and constructed,
but at minimum it will include a basic meteorology package capable of obtaining
measurements throughout the duration of the flight.
PHY420 students will have responsibility for ensuring that
timelines set by the team are met and that deliverables are completed on schedule.
ESSC101 and PHY379 students will
meet regularly with the PHY420 students (at minimum once per week) to coordinate work on the
project, learn about relevant safety procedures, and establish standard operating procedures.
An initial test launch will be scheduled for April/May 2023.
We anticipate that the NEBP supply materials sometime in May. If
delivery of these materials occurs after the 1st test flight, we will work with
colleagues at neighboring University of Michigan to borrow necessary equipment
to proceed with the test flight (this was done from 2012 - 2016). By utilizing existing courses
at EMU in this manner, we anticipate being able to "hit the ground running" by the
when NEBP materials are delivered in May 2023.

In May 2023, Prof. Pawlowski will attend the regional pod workshop with one of
the student leaders from PHY379 (2 potential students have already been approached
for this role).
As PHY420 students graduate in May 2023, the PHY379 and ESSC101 students will be
asked to take over leadership of the project. We plan to provide
summer funding for up to 3 physics students and 3 ESSC students to make modifications to the
payload during Summer 2023 based on the results of the spring test flight/receipt
of equipment from NEBP.
We will then schedule a 2nd local test flight with the modified payload and
NEBP supplies for September
2023.

 During the Fall 2023 semester, all EMU APEX students will enroll in PHY379 in order
 to prepare for the October 2023 eclipse, attend the event, and perform post flight
 analysis. During the Winter 2024 semester, all participating students will
 take either PHY420 or PHY379. This cohort of students
will be involved with the project from program onset in January 2023 through initial analysis
of the April 2024 eclipse. We anticipate this
team will be comprised of 5 Physics students and 5 ESSC101 students for a total
of 10 students participating in the two eclipse campaigns. Throughout the
project, we will continue to recruit students from subsequent ESSC101 sections as well as
other physics students to participate in the long term goals of the project. While students that
join EMU APEX after the initial cohort may have missed the opportunity to
attend the eclipse campaigns, they can be involved in the data analysis
that results from those events and assist with dissemination activities.

Since Profs. Pawlowski and Kovacs will be the
instructors for all courses relevant to the NEBP, they will be able to support the
team, as part of their teaching load, from the beginning of the project until
closeout. They will both attend the two eclipse campaigns and at least one
of the faculty mentors will be onsite for all test flights. Additionally,
they anticipate collectively dedicating at minimum 6 hours per week to working with
the students on activities related to the NEBP during the academic year throughout
the duration of the program and will be available as needed to advise students contributing
to the project during Summer 2023.

\subsection{Anticipated Outcomes}
While in many cases, the learning outcomes for both groups of students will be similar, there are
some differences that make this approach particularly compelling. First, this
project would serve all participating students by providing an opportunity to develop
their scientific analysis, design, and laboratory skills in the context of a long-term project. The students will have
personal responsibility for the success of the missions, while having
an established support network via the faculty mentors and NEBP training opportunities and resources.
A HAB project is ideal for giving students an opportunity to develop their scientific and laboratory skills.
Unlike in many lecture and lab courses, where students are given a set of instructions
to follow, that if followed precisely, will result in success (measured by obtaining the correct answer
or result), this program will require that students step outside their comfort zone and work together to
complete a project for which there is no set "correct answer" and where completing all the steps to the
best of their ability does not necessarily guarantee success.

The physics students will be tasked with
applying the skills that they have developed over 2 or 3 years of theoretical and laboratory courses.
In particular they will gain valuable skills in the enginnering design process, really learn the importance
of testing, develop their electronic skills, build their understanding of integrated systems and data analysis,
 and learn
about the processes that shape the atmosphere. Additionally, PHY420 students will
get to build their mentorship skills as they are asked to serve
as mentors for participating students from ESSC101 and PHY379.

The ESSC 101 students will use knowledge gained
in their Weather and Forecasting course to contribute to the design of the payload and help educate
the physics students on atmospheric processes. They will learn about
the construction of scientific instrumentation, develop basic electronics skills, laboratory skills,
and learn how to perform analysis of data obtained by their own instrumentation.

\subsection{Team Experience}
There is not currently an active high altitude balloon program at EMU. However, both faculty mentors have
prior experience working with atmospheric data in a relevant context. Specifically,
Prof. Kovacs’ research deals with satellite-based remote sensing of the atmosphere using spaceborne lidar and visible and infrared emission imaging.
Prof. Pawlowski's expertise is in analysis of planetary atmospheres using observations
and computer models. Additionally, from 2012 - 2016 he
mentored students as part of his Physics
Capstone course as they designed, built and launched high altitude weather balloons (using
equipment borrowed from colleagues at the University of Michigan). One of the
limitations of working only with Capstone students
was that these students were all seniors which meant that there was very little continuity from year to year.
The proposed project is particularly compelling to both mentors
because it would establish a long term collaborative
stratospheric ballooning program at EMU that would see students be able to
participate in the program throughout their undergraduate career.

Given that EMU doesn't have an active ballooning program, but both faculty mentors
do have relevent experience, it is expected that, with faculty assistance, students involved in EMU APEX will be able to learn
and participate at a high level given training workshops and resources on safety and operations procedures, flight regulations,
procedures surrounding helium acquisition and transportation, etc. provided by the NEBP.
They should not require additional external support beyond that provided by NEPB.
While it is not necessary, we would be willing to be paired with a team from
another institution.



\newpage
\section{Curricula Vitae}
\includepdf[pages=-]{cvtwopage_NEBP.pdf}


\newpage

\section{Financial Resources}
\vspace{-.15in}
In order to evaluate the feasibility of obtaining funds to support travel,
helium and stipend costs associated with this project, we
assumed a minimum of 10 travelers to both eclipse events, 2 travelers
to attend the regional training, and 4 helium fills. Accounting for
flights to the Oct. 2023 eclipse, lodging, mileage and per diem expenses we
arrived at an approximate travel budget of \$26,000
plus an additional \$1,000 for helium fills.
We plan to hire up to
6 students to support the project in the Summer 2023 semester (\~ 60 hours each)
for a cost of \$5,000.
Thus, the total projected cost is \$32,000.

\vspace{-.1in}
\subsection{Sources of Support}
\vspace{-.1in}
\begin{wraptable}{r}{11cm}
  \vspace{-.2in}

  \centering
  \begin{tabular}{|l|r|}
  \hline
  Source&Amount\\
  \hline
  Physics and Astronomy- General*&\$5000\\
  Physics and Astronomy- lab and equipment*&\$1000\\
  Geography and Geology- General*&\$5000\\
  MSGC 2022 - 2023&\$5000\\
  MSGC 2023 - 2024&\$5000\\
  URSP&\$2200\\
  Pawlowski Student Support*& \$3000\\
  URSP Travel and Equipment&\$600\\
  Dean Faculty Travel Grant&\$1000\\
  Anticipated University contribution&\$5000\\
  \hline
  {\bf Total}&{\bf \$32,800}\\
  \hline
  \end{tabular}
  \caption{\small Anticipated sources of financial support for travel and equipment.
  * indicates funds are already committed.}
  \label{money}
  \vspace{-.2in}
  \end{wraptable}
A summary of anticipated funding is provided in Table \ref{money}.
EMU is reliant on State funding and tuition dollars to cover expenses. It is
not typical to have large discretionary accounts for supporting significant efforts.
So, we take a piecemeal
approach towards funding for the NEBP. Note that many of the funding sources will
require submission of an application after NEBP selections are made before funds are fully
committed.

We have been given approval for initial funds in support of this effort
from the Physics and Astronomy and
Geology and Geography Departments.
We have also contacted the Michigan Space Grant Consortium.
While they have not yet allocated funds specifically for the NEBP, they
have advised us to apply for a Hands-On NASA-oriented Experiences for Student Groups (HONES)
grant and Research Seed grant. We will apply for
a HONES awards by November 15, 2022 and then a subsequent MSGC award in 2023.

We will apply for to EMU's Undergraduate
Research Stimulus Program (URSP) for part of the summer student funding. The remainder
of  student funding will be provided by
Prof. Pawlowski's development account for this purpose.
Helium costs will be covered by the Department of Physics and Astronomy's laboratory
and equipment account. During the 2023 - 2024 academic year both faculty mentors will
apply for a Travel Grant from the College of Arts and Sciences Dean.

Finally, we are in discussions with the EMU Foundation, the College of Arts and Sciences
Deans Office, and EMU Provost's office regarding securing additional funding to cover
any additional expenses, including travel for additional team members. As the bulk of the expenses related to this project will occur
during the 2023 - 2024 fiscal year, it is difficult for these offices to make firm budgetary
commitments at this time, but we are confident that we will be able to obtain at minimum
\$5000 between them.






\newpage
\section{Student Support}
All participating students will receive credit towards their degree for
participating in the NEBP. During the Winter 2023 semester, all participating students
will be enrolled in PHY420 (4 credit hours (CH)) and PHY379 (2 CH)
or ESSC101 (4 CH). PHY420 students will graduate after the the Winter2023 semester and
turn over leadership of the project to the other students. They are not expected to
 participate in the eclipse campaigns.
All participating students will enroll in PHY379 during the Fall 2023 term. During the
Winter 2024 term, participating physics students will enroll in PHY420. We will run
another section of PHY379 that participating students may opt to register for if they
choose to.

A significant portion of the payload build will be completed during the Winter 2023
term. Following the spring test flight, modifications to the payload will be
performed during the Summer 2023 term. We currently project enough funding to
provide a stipend for 6 students, though we will continue to seek funding to support
additional students.




\section{Physical Spaces}
The Physics and Astronomy Department at EMU, located in Strong Hall, has a dedicated
laboratory for our Capstone classes that includes basic electronics equipment,
soldering stations, ample storage and space for teams to work on various projects.
 The NEBP team would be given their own space in this laboratory for the duration of
 the project. Additionally, our Department has a dedicated student Maker Space that all
  team members would have access to as well as a staffed machine shop if their payloads
  require custom parts.


\section{Letters of Support}
\includepdf[pages=-]{deansupport.pdf}
\includepdf[pages=-]{behringerletter.pdf}

\section{DEIA Plan}

Improving diversity and addressing equity and inclusion in the classroom
at Eastern Michigan University has been a priority for several years, formally 
begining with the creation of the Department of Diversity and Community 
Involvement (DCI). The DCI offers many opportunities for students to learn, 
develop and find a place to belong through various programs and services. 
The DCI aims to assist students with their transition to college, 
engage them in meaningful dialogue, help them develop their leadership potential, 
and equip them with the tools to become responsible global citizens. 
We encourage students to ask difficult questions, challenge the status quo,
 and broaden the way they view themselves, the world, and our campus community. 
 
The student
body at EMU has traditionally been more diverse than similar institutions. Based
on the most recent data from the Fall 2021 semester, 62\% of the student body
identify as female and 32\% as minority, with 33\% of all students being a
recipient of a Pell Grant. Yet, the STEM
programs at EMU
continue to face challenges attracting underrepresented populations like many other
institutions throughout the country.

The College of Arts and Sciences, and in
particular the Department of Physics and Astronomy, have started to take steps to address these issues. As a Department, we are currently
participating in the AAAS Sea Change program with the goal to begin to transform
our programs in support of a more diverse and inclusive learning environment. Additionally,
Prof. Pawlowski, along with other STEM peers in the College of Arts and Sciences, is currently
participating in the Inclusive STEM Teaching Project and an associated learning community
at EMU.
The NEBP presents us with a compelling program that can be used as a tool
to improve engagement and a sense of community among underrepresented
groups. In order to accomplish this, we designed
the program to bring together upper level undergraduate
physics and astronomy students (a population that
tends to be represented by white males and females)
to work with younger students that are recruited from General Education coursework.
These courses, which include ESSC101, typically have enrollments that are more representative
of the diverse student body as a whole than the Physics or Geography and Geology Departments
themselves.




\section{Sustainability Plan}

As mentioned in Section \ref{intro}, one of the main goals of this project
is to establish a long term stratospheric ballooning program at EMU. Equipment
obtained as part of the NEBP as well as the initial momentum created by completing two eclipse campaigns will make is possible to launch
EMU APEX on a trajectory to become a sustainable program on EMU's campus. In informal
discussion with students, we know that there is strong support for
being a part of creating this program. Throughout the duration of the
NEBP, we plan continuously recruit students to EMU APEX by showcasing
milestones and results in introductory courses (including subsequent sections
of ESSC101 as well as sections of our introductory mechanics courses). While
students recruited to EMU APEX after
the initial cohort of participants may not be able to attend the eclipse campaigns,
they will have the opportunity to participate in
analysis of the eclipse events (and other future launches) as well as assist with manuscript and presentation
preparation.

We plan to allocate space permanently to EMU APEX in our Capstone lab
so that future teams have continued access to the appropriate tools and
equipment. After conclusion of the NEBP, we foresee requiring a relatively
small yearly budget (~\$3,000) to replace consumables (primarily helium and balloons)
and to cover travel expenses for local/regional launches
to sustain EMU APEX. There are a variety of potential sources for funding, including
the MSGC and the EMU Foundation. Also, we plan to work with the Accounting Department
to establish an endowed account and to use the
eclipse events to initialize a targeted donation campaign in support of
the creation of EMU APEX.



\section{Team Assurances}











\end{document}
