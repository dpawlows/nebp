\documentclass[12pt]{article}

%%%%%%%%%%%%%%%%%%%%%%%%%%%%%%%%%%
%%
%% Preamble
%%
%%%%%%%%%%%%%%%%%%%%%%%%%%%%%%%%%%

\usepackage{epsfig}
\usepackage{graphics}
% \usepackage{color}
\usepackage[square]{natbib}
%\usepackage{helvet}
%\usepackage{pslatex}
%\usepackage{times}
%\usepackage{palatcm}
\usepackage[sc]{mathpazo}
\usepackage{wrapfig}
\usepackage{pawlowski}
\usepackage{sectsty}


%\setcounter{secnumdepth}{0}

\setlength{\oddsidemargin}{0 in}
\setlength{\evensidemargin}{0 in}
\setlength{\topmargin}{-.20 in}
\setlength{\textwidth}{6.5 in}
\setlength{\textheight}{9.0 in}

%\renewcommand{\sfdefault}{phv}
%\renewcommand{\rmdefault}{phv}


\begin{document}

\Large
\begin{center}
The EMU Atmospheric Physics EXploration (EMU APEX) Program\\
\large
\vspace{2cm}
Nationwide Eclipse Ballooning Project\\
{\bf Engineering Track}
\vspace{1in}

\normalsize
\begin{table}[h]
\centering

  \begin{tabular}{ll}
  David Pawlowski&Thomas Kovacs\\
  Professor of Computational Physics&Professor of Meteorology\\
    Physics and Astronomy Dept.&Geography and Geology Dept.\\
  dpawlows@emich.edu&tkovacs@emich.edu\\
  240D Strong Hall&140 H Strong Hall\\
  734-709-0311&734-487-8591

  \end{tabular}
\end{table}


Eastern Michigan University\\
 R2 Research University\\
Ypsilanti, MI 48197
\end{center}

\small\normalsize
\newpage
\tableofcontents

\newpage
\section{Team Mentoring Approach Plan}
The primary objective of this project is to form an interdisciplinary
high altitude weather balloon (HAB) program at Eastern Michigan University (EMU).
The first cohort of students will design, construct, and launch a
high weather altitude balloon payload
capable of taking measurements during 2 upcoming solar ecplise events and subsequently
establish a long-term stratospheric ballooning program. We see this as an opportunity to build a sense of
community between the students across the College of Arts and Sciences at EMU and invision a program
that leverages the expertise of upper level science students from the Department of Physics and Astronomy
to work alongside students from introductory courses (initially, ESSC101:
Introduction to Weather and Forecasting). By participating in the {\bf Engineering Track}, we look to
provide the participating students and opportunity to get their hands dirty by taking part in the
engineering design process, designing and building scientific instrumentation capable of making
meaningful contributions to the understanding of our environment.

\subsection{Project Overview and Timeline}
Our priority for this project is to begin an ongoing high altitude ballooning program
at EMU that provides students from a diverse set of interests and experience levels
the opportunity to participate in a long-term, hands on, real world project. As such,
we plan to initially recruit students from 2 student populations and incorporate
activities related to the HAB project in existing courses: PHY420- Physics
Capstone and ESSC101: Introduction to Weather and Forecasting.
Additionally, most of the PHY420 students will graduate in May or December 2023, so in order to ensure continuity
with the project, we will offer a special topics course (PHY379) that Junior and Sophomore
level participants will take. This course is necessary due to prerequiste requirements and other
course requirements for PHY420 students.


Prof. Pawlowski is the instructor for PHY420. As part of this course, students are tasked with completing
a semester long research project. Encorporating stratospheric ballooning into this course is a
natural fit (and HAB projects have been done as part of PHY420 in previous years). Prof. Kovacs is the instructor for ESSC101. During the Winter 2023 semester, under the guidance of PHY420 students,
 ESSC101 and PHY379 students will learn about the design, construction and operation
of a high altitude weather balloon in order to prepare them for taking ownership of the
ballooning program after the semester ends. ESSC101 and PHY379 students will
participate in the design and build of the initial payload with assistance from the PHY420 students.

All students will work collaboratively to establish important milestones and to identify
and assign roles within the project, based on NEBP website recommendations and requirements.
The team will collectively decide on the specific scientific payload to be designed and constructed,
but at minimum it will include a basic meterology package capable of obtaining
measurements throughout the duration of the flight.
PHY420 students will have responsibility for ensuring that
timelines set by the team are met and that deliverables are completed on schedule.
ESSC101 and PHY379 students will
meet regularly with the PHY420 students (at minimum once per week) to coordinate work on the
project, learn about relevant safety proceedures, and establish standard operating proceedures.
An initial test launch will be scheduled for Aptil/May 2023 and all students will be invited
to participate. We anticipate that the NEBP supply materials sometime in May. If
delivery of these materials occurs after the 1st test flight, we will work with
colleagues at the neighboring University of Michigan to borrow necessary equipment
to proceed with the launch, as we have done in the past. By utilizing existing courses
at EMU in this manner, we anticipate being able to "hit the ground running" by the
time online training materials are available from NEBP in April 2023 and
materials are delivered in May 2023.

In May 2023, Prof. Pawlowski will attend the regional pod workshop with one of
the student leaders from PHY379 (Pawlowski has already identified 2 potential students
for this role).
As PHY420 students graduate in May 2023, the PHY379 and ESSC101 students will be
asked to take over leadership of the project. We are in the process of applying for funds
to support a limited amount of summer funding for students to make modifications to the
payload during Summer 2023 based on the results of the spring test flight/receipt
of equipment from NEBP. We anticipate
being able to support up to 6 students during the summer in this role for a total of 40 hours
in the lab. This amount of summer effort is appropriate since a significant portion of the scientific payload build will take place during the
Winter 2023 semester.
We will then schedule a 2nd local test flight with the modified payload and
NEBP supplies for September
2023.

 During the Fall 2023 semester, all participating students will enroll in PHY379 in order
 to prepare for the October 2023 ecplise, attend the event, and perform post flight
 analysis. During the Winter 2024 semestser, the project leaders from the physics department
will be ready to take PHY420. Therefore that cohort of students
along with the original recruits from ESSC101 will be involved with the project from program onset in January 2023 through final analysis of the April 2024 eclipse. We anticipate this
team will be comprised of 5 Physics students and 5 ESSC101 students for a total
of 10 students participating in the two eclipse campaigns.

Throughout the program, we plan continuously recruit students to EMU APEX throughout
 the NEPB program
so that stratospheric ballooning at EMU can continue beyond project closeout. Students recruited after
the initial cohort of participants will have the opportunity to paticipate in
analysis of the eclipse events as well as well as assist with manuscript and presentation
preparation.

Since Profs. Pawlowski and Kovacs will be the
instructors for all courses relevant to the NEBP, they will be able to support the
team, as part of their teaching load, from the beginning of the project until
closeout. They will both attend the two eclipse campaigns and at least one
of the faculty mentors will be onsite for all test flights. Additionally,
they anticipate collectively dedicating at minimum 6 hours per week to working with
the students on activities related to the NEBP during the academic year throughout
the duration of the program and will be available as needed to advise students contributing
to the project during Summer 2023.

\subsection{Anticipated Outcomes}
While in many cases, the learning outcomes for both groups of students will be similar, there are
some differences that make this approach particularly compelling. First, this
project would serve all participating students by providing an opportunity to develop
their scientific analysis, design, and laboratory skills in the context of a long-term project. The students will have
personal responsibility for the success or failure of the missions, while having
an established support network via the faculty mentors and NEBP training opportunities and resources.
A HAB project is ideal for giving students an opportunity to develop their scientific and laboratory skills.
Unlike in many lecture and lab courses, where students are given a set of instructions
to follow, that if followed precisely, will result in success (measured by obtaining the correct answer
or result), this program will require that students step outside their comfort zone and work together to
complete a project for which there is no set "correct answer" and where completing all the steps to the
best of their ability does not necessarily guarantee success.

The physics students will be tasked with
applying the skills that they have developed over 2 or 3 years of theoretical and laboratory courses.
In particular they will gain valuable skills in the enginnering design process, really learn the importance
of testing, develop their electronic skills, build their understanding of integrated systems and data analysis,
 and learn
about the processes that shape the atmosphere. Additionally, the physics students will
get to build their mentorship skills as they are asked to serve
as mentors for participating students from ESSC101 and PHY379.

The ESSC 101 students will use knowledge gained
in their Weather and Forecasting course to contribute to the design of the payload and help educate
the physics students on atmospheric processes. They will learn about
the construction of scientific instrumentation, develop basic electronics skills, laboratory skills,
and learn how to perform basic data analysis using their own data.

\subsection{Team Experience}
There is not currently an active high altitude balloon program at EMU. However, both faculty mentors have
prior experience working with atmospheric data in a relevant context. Specifically,
Prof. Kovacs' research deals with remote sensing of the atmosphere using a variety of technologies.
Additionally, while Prof. Pawlowski's expertise is in planetary physics, from 2012 - 2016 he
mentored students each year as part of his Physics
Capstone course as they designed, built and launched high altitude weather balloons (using
equipment borrowed from colleagues at the University of Michigan). One of the
limitations of working only with Capstone students
was that these students were all seniors which meant that there was very little continuity from year to year.
The proposed project is particularly compelling to Prof. Pawlowski
because it would establish a long term collaborative
stratospheric ballooning program at EMU that would see students become involved in the program
early in their undergraduate career setting them up to become leaders of the program in subsequent years.

Given that EMU doesn't have an active ballooning program, but both faculty mentors
do have relevent experience, it is expected that, with faculty assistance, students involved in EMU APEX will be able learn
and participate at a high level given training workshops and resources on safety and operations procedures, flight regulations,
procedures surrounding helium acquisition and transportation, etc provided by the NEBP.
The should not require additional external support beyond that provided by NEPB.
While it is not necessary, we would be willing to be paired with a team from
another institution.



\newpage
\section{CVs}



\newpage

\section{Financial Resources}
In order to evaluate the feasiblity of obtaining funds to support travel,
helium and stipend costs associated with this project, we
assumed a minimum of 10 travelers to both eclipse events, 2 traveler
to attend the regional training, and 4 helium fills. For
each trip we assumed 6 nights of accomodations would be requried with
double occupancy. While we anticipate booking round trip flights for the majority
of participants of the
October 2023 event (which is far from EMUs campus), we plan
to drive to the April 2024 event. For the regional training workshop,
we assumed 3 nights of accomodations and only per diem and milage expenses.
With these assumptions, we reached a minimum travel budget of approximately \$26,000
plus an additional \$1,000 for helium fills for a total of \$27,000.

Additionally, we plan to hire up to 6 students to support the project in a limited role
during the Summer 2023 semester. EMU has a program for summer research
support (the Undergraduate Research Stimulus Program (URSP) that
can support 1 student.
We will apply for one of these awards to support the primary
student leader during the summer to ensure program continuity. We have budgeted
funding to support the other 5 students in a limited role of 10 hours per week
for 5 weeks. The total anticipated cost of student support based on these assumptions is \$5,500.

The total projected cost for travel, helium and student stipends is \$32,500.

\subsection{Sources of Support}

We have been given approval for initial funds in support of this effort
from the Department of Physics and Astronomy, the Department of
Geology and Geopraphy, and the College of Arts and Sciences Deans office.
Additionally, we have contacted the Michigan Space Grant Consortium
and while they have not yet allocated funds specifically for the NEBP, they
have advised us to apply for a Hands-On NASA-oriented Experiences for Student groups
(HONES) grant and/or a Research Seed grant.
Applications for the HONES programs are accepted on a rolling basis and seed
this project would be eligable for an award during both the 2022-2023 and
2023-2024 fiscal years. Profs Pawlowski and Kovacs will apply for
the a HONES awards by November 15, 2022.

As mentioned, funds from EMU's URSP program will support the primary student leader while
Prof. Pawlowski will utilize existing summer research
student funds as part of his professional development account to support up to 5 students.
Helium costs will be covered by the Department of Physics and Astronomy's laboratory
and equipment account.


 See Table \ref{money}.


\begin{table}[h]
\centering
\begin{tabular}{|l|r|}
\hline
Source&Amount\\
\hline
Physics and Astronomy- General&\$5000\\
Physics and Astronomy- lab and equipment&\$1000\\
Geography and Geology- General&\$5000\\
College of Arts and Sciences&\$5000\\
MSGC 2022 - 2023&\$5000\\
MSGC 2023 - 2024&\$5000\\
URSP + Pawlowski Student Support& \$5500\\
URSP Travel and Equipment&\$600\\

\hline
{\bf Total}&{\bf \$32,100}\\
\hline
\end{tabular}
\caption{Current sources of financial support for travel and equipment}
\label{money}
\end{table}


Additionally, we've been talking with the EMU Foundation. They are throwing around
crazy numbers....

\newpage
\section{Student Support}
All participating students will receive credit towards their degree for
participating in the NEPB. During the Winter 2023 semester, all participating students
will be enrolled in PHY420 (4 credit hours (CH), PHY379 (2 CH) and/or ESSC101 (4 CH).
Additionally, all participating students will have the option to enroll again in
additional sections of PHY379 during the Fall 2023 term. The Special Topics
course will not have any prerequisites, so registration will be open to
students from any major.

A significant portion of the payload build will be completed during the Winter 2023
term. Following the spring test flight, we anticipate modifications to the
payload will be desired. Therefore, we plan to recruit up to 6 students
to work on the payload in a limited manner (approximately 40 hours for each
student). These students will be provided with a
stipend for their time spent on the project.


\section{Physical Spaces}
The Physics and Astronomy Department at EMU, located in Strong Hall, has a dedicated laboratory for our Capstone classes that includes basic electronics equipment, soldering stations, ample storage and space for teams to work on various projects. The NEBP team would be given their own space in this laboratory for the duration of the project. Additionally, our Department has a dedicated student Maker Space that all team members would have access to as well as a staffed machine shop if their payloads require custom parts.


\section{Letters of Support}


\section{DEIA Plan}



\section{Sustainability Plan}

\section{Team Assurances}











\end{document}
